\documentclass{article}
\usepackage{amsmath}
\usepackage{amssymb}
\usepackage{algorithm}
\usepackage{algorithmic}
\usepackage{amsfonts}

\begin{document}

\title{Algoritmo de Programação Dinâmica para o Probelma do Caminho Mínimo}
\author{}
\date{}
\maketitle

\section*{1. Introdução}
O algoritmo em questão é uma aplicação da equação de Bellman-Ford para encontrar o caminho mínimo em um grafo. A abordagem usada é a Programação Dinâmica, onde buscamos o menor custo para percorrer de um nó inicial \( s \) a um nó final \( t \) em um grafo com \( n \) nós.

\section*{2. Definições e Configurações Iniciais}
Seja \( G = (V, E) \), onde \( V \) é o conjunto de nós e \( E \) o conjunto de arestas com os respectivos custos. O algoritmo usa os seguintes componentes:

\begin{itemize}
    \item \( n \): número de nós.
    \item \( D \in \mathbb{R}^{n \times n} \): matriz de adjacência que representa os custos das arestas entre os nós, onde \( D_{ij} \) indica o custo da aresta de \( i \) para \( j \). Se não existir aresta direta entre os nós, usamos \( \infty \) para representar essa ausência.
    \item \( s \): nó de origem.
    \item \( t \): nó de destino.
    \item \( J \in \mathbb{R}^{1 \times n} \): vetor de custos mínimos para alcançar cada nó, inicializado com \( J[s] = 0 \) e \( J[v] = \infty \) para todos os \( v \neq s \).
    \item \( P \in \mathbb{Z}^{1 \times n} \): matriz de precedência para rastrear o caminho.
\end{itemize}

\section*{3. Inicialização}
Inicializamos os vetores \( J \) e \( P \):
\[
J[v] = 
\begin{cases}
0 & \text{se } v = s \\
\infty & \text{caso contrário}
\end{cases}
\]
e
\[
P[v] = 0 \quad \forall v \in V
\]

\section*{4. Equação de Bellman-Ford}
A equação de Bellman para atualizar os custos mínimos é dada por:
\[
J[v] = \min \left( J[v], J[u] + D[u,v] \right)
\]
onde \( u \) é um nó anterior e \( v \) o nó atual. Se o custo para alcançar \( v \) a partir de \( u \) for menor do que o custo já conhecido, então atualizamos o custo \( J[v] \) e a matriz de precedência \( P \).

\section*{5. Passo a Passo do Algoritmo}

\subsection*{5.1. Laços de Iteração}
O algoritmo realiza \( n-1 \) iterações, onde \( n \) é o número de nós. Durante cada iteração, percorremos todos os pares de nós \( (u, v) \) para verificar se uma atualização é necessária.

\begin{algorithm}
\caption{Algoritmo do Caminho Mínimo}
\begin{algorithmic}
\STATE \textbf{Para cada iteração } $iter$ \textbf{ de } $1$ \textbf{ a } $n-1$:
\STATE \quad \textbf{Para cada nó } $v$ \textbf{ de } $1$ \textbf{ a } $n$:
\STATE \quad \quad \textbf{Para cada nó } $u$ \textbf{ de } $1$ \textbf{ a } $n$:
\STATE \quad \quad \quad \textbf{Se } $D[u, v] \neq \infty$ \textbf{ e } $J[v] > J[u] + D[u, v]$ \textbf{ então }
\STATE \quad \quad \quad \quad $J[v] \leftarrow J[u] + D[u, v]$
\STATE \quad \quad \quad \quad $P[v] \leftarrow u$
\end{algorithmic}
\end{algorithm}

\subsection*{5.2. Impressão do Custo Mínimo}
Após todas as iterações, o valor \( J[t] \) representa o custo mínimo do caminho de \( s \) a \( t \).
\[
\text{Custo mínimo da rota: } J[t]
\]

\subsection*{5.3. Rastreamento do Caminho Mínimo}

\subsubsection*{Objetivo do Rastreamento}
Depois de calcular os custos mínimos para todos os nós, o vetor de custos \( J \) fornece o custo mínimo para alcançar o nó \( t \) a partir de \( s \). No entanto, para conhecer a sequência de nós que compõem o caminho com esse custo mínimo, é necessário rastrear os nós intermediários que levam do nó final \( t \) ao nó inicial \( s \).

\subsubsection*{Função da Matriz de Precedência \( P \)}
A matriz de precedência \( P \) é utilizada para rastrear o nó anterior em cada atualização do custo mínimo. Quando o algoritmo verifica se é possível melhorar o custo para alcançar um nó \( v \) a partir de um nó \( u \) usando a equação de Bellman:
\[
J[v] = \min \left( J[v], J[u] + D[u, v] \right),
\]
se essa atualização resultar em um custo menor, então a matriz de precedência é atualizada para refletir que o melhor caminho para \( v \) passa por \( u \):
\[
P[v] = u.
\]
Isso significa que o nó \( u \) precede o nó \( v \) no caminho de custo mínimo.

\subsubsection*{Como o Rastreamento Funciona}
O rastreamento começa no nó de destino \( t \) e utiliza a matriz de precedência \( P \) para encontrar o nó anterior que leva a \( t \). Esse processo é repetido até que o nó de origem \( s \) seja alcançado. O passo a passo é o seguinte:

\subsubsection*{Inicialização}
O algoritmo inicia o rastreamento definindo o nó atual como \( v = t \), que é o nó de destino. A rota inicial é então definida como:
\[
\text{rota} = [t],
\]
pois começamos do final.

\subsubsection*{Laço de Rastreamento}
Enquanto o nó atual \( v \) não for igual ao nó de origem \( s \), continuamos retrocedendo:
\begin{itemize}
    \item Atualizamos \( v \) para o nó que precede o nó atual no caminho mínimo, isto é, \( v = P[v] \).
    \item Adicionamos esse novo nó \( v \) ao início do vetor de rota.
\end{itemize}

\subsubsection*{Finalização}
O laço termina quando \( v \) é igual ao nó de origem \( s \), indicando que alcançamos o início do caminho. O vetor \( \text{rota} \) agora contém todos os nós do caminho ótimo, desde \( s \) até \( t \).

\subsubsection*{Exemplo para Esclarecimento}
Suponha que temos a matriz de precedência \( P \) preenchida da seguinte forma após as iterações:
\[
P = [0, 1, 2, 3, 2].
\]
Isso indica que:
\begin{itemize}
    \item Para o nó 2, o nó precedente é 1.
    \item Para o nó 3, o nó precedente é 2.
    \item Para o nó 4, o nó precedente é 3.
    \item Para o nó 5, o nó precedente é 2.
\end{itemize}

Se o nó de destino for 5 e o nó inicial for 1, o rastreamento se dará da seguinte forma:
\begin{enumerate}
    \item Começamos com \( v = 5 \), então a rota inicial é \( [5] \).
    \item O nó que precede 5 é 2 (ou seja, \( P[5] = 2 \)), então a rota agora é \( [2, 5] \).
    \item O nó que precede 2 é 1 (ou seja, \( P[2] = 1 \)), então a rota agora é \( [1, 2, 5] \).
\end{enumerate}
O rastreamento termina, pois chegamos ao nó inicial \( s = 1 \).

\subsubsection*{Importância da Matriz de Precedência}
A matriz de precedência \( P \) armazena a informação sobre quais são os nós predecessores no caminho de custo mínimo. Essa matriz é essencial para reconstruir o caminho ótimo, pois indica a direção que devemos seguir ao retroceder desde o nó final \( t \) até o nó inicial \( s \). Sem essa matriz, não seria possível determinar o caminho seguido, apenas o custo mínimo total.

Portanto, a matriz de precedência serve para registrar o "mapa" do caminho, enquanto o vetor de custos \( J \) fornece o "preço" para percorrer esses caminhos.


Para encontrar o caminho seguido, utilizamos a matriz de precedência \( P \), rastreando o caminho de volta a partir do nó \( t \).
\begin{algorithm}
\caption{Rastreio do Caminho}
\begin{algorithmic}
\STATE $v \leftarrow t$
\STATE $\text{Rota} \leftarrow [v]$
\WHILE{$v \neq s$}
\STATE $v \leftarrow P[v]$
\STATE $\text{Rota} \leftarrow [v, \text{Rota}]$
\ENDWHILE
\end{algorithmic}
\end{algorithm}

\section*{6. Resultado}
O custo mínimo para o caminho de \( s \) a \( t \) é dado por \( J[t] \), e o caminho pode ser rastreado usando o vetor de precedência \( P \).

\section*{7. Considerações Finais}
O algoritmo de Bellman permite encontrar o caminho mínimo em grafos, mesmo quando existem ciclos negativos, desde que esses ciclos não sejam acessíveis a partir do nó de origem \( s \). A complexidade do algoritmo é \( O(n^3) \), devido aos três laços aninhados.

\end{document}

